\part{整関数}
高校数学では$\sin x$,$e^x$の微分を図形から設定した.
しかし厳密には弧長等含め,きちんと定義する必要がある.
そのためここでは,$\sin x, e^x$を収束べき級数として定義する.
ここではべき級数の理論を実際に紹介する.

\subsection{無限級数の収束}
\subsection{関数の無限和の収束}
\subsection{べき級数の収束とその判定方法}
\begin{screen}
\begin{dfn}
\begin{equation*}
 f(x) := \sum_{n=0}^{\infty} a_n(x -a)^n
\end{equation*}
を$a$を中心とする\textbf{べき級数}という.
\end{dfn}
\end{screen}

\begin{lem}
  以下を満たす$R$が一意に定まる.
  \begin{itemize}
    \item $|x -a | < R$に対し,$f(x)$はwell-defined.つまり,絶対収束する.
    \item $|x - a| >R$ならば,発散する.
  \end{itemize}
この時,上で定める$R$を一様収束半径という.
\end{lem}

\subsection{指数関数,三角関数}
\begin{screen}
\begin{dfn}
 \begin{equation*}
  e^x := \sum_{n=0}^{\infty} \frac{1}{n!}x^n
 \end{equation*}
\end{dfn}
\end{screen}
とする.
