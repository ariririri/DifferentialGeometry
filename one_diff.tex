\part{一変数の微分}
ここでは,一変数関数の微分に関する基本的な性質を述べる.
定義から始まり、基本的な性質を調べる.
代数的な性質,位相的な性質,合成関数の微分法,積の微分法を説明し,
それらを元に初等的な関数の微分を記述する.
\section{微分の初等的な性質}

\begin{dfn}[微分]
ある関数$f:U \to \mathbb{R}$に対し,
$x=a$で\textbf{微分可能}とは,
\begin{equation*}
\lim \frac{f(x) - f(a)}{x-a}
\end{equation*}
が存在することである.
上の極限を$f'(a)$とかき,$f(x)$の$x=a$での\textbf{微分}という.
極限の存在はある値$b$が存在し,$\forall \epsilon > 0$に対しある$\delta > 0$をうまく取ると$|x - a| < \delta$となる全ての$x$に対し,$\frac{f(x) - f(a)}{x-a} - b| < \epsilon$が成り立つことを意味する.
\end{dfn}

上の定義は関数が一点で微分可能かどうかを定義したものである.
つまり,上の定義では関数と点の組に対して,微分可能かを判定させる.
では関数に対して微分可能を定義するにはどうすればいいか.
一つの自然な答えは定義域の全ての点で微分可能である.
これを具体的に記述する.
\begin{dfn}
関数$f$が定義域$U$上の任意の点で微分可能な時,$f$を\textbf{微分可能}という.
\end{dfn}


実際にどのような関数が上の定義を満たしているかをみる.


\begin{epl}
以下の関数は微分可能である.
\begin{equation*}
f(x) = x^n (n \ge 0)
\end{equation*}
実際に微分の定義に従い,極限を求める.
\begin{equation*}
\frac{f(x) - f(a)}{x-a}= \sum_{i=0}^{n-1} x^{n-1-i}a^i
\end{equation*}
より,$g(x):=\frac{f(x) - f(a)}{x-a}$とすると,$g(x)$は連続関数であり,$g(a) = na^{n-1}$となる.
よって
\begin{equation*}
\lim_{x \to a} g(x) = na^{n-1}
\end{equation*}
となり,微分可能である.
\end{epl}


もう一つ微分可能な例をみる.

\begin{epl}
$f(x)$ = $
\begin{cases}
x^3 & (x \ge 0)\\
x^2 & (x < 0)
\end{cases}
$は微分可能.
上の例から,$x \neq 0$では微分可能である.
$a = 0$では

\begin{equation*}
g(x) := \frac{f(x) -f(0)}{x - 0} =
\begin{cases}
x^2 & (x \ge 0)\\
x & (x < 0)
\end{cases}
\end{equation*}

となるので,$\displaystyle \lim_{x \to 0} g(x) = 0$となり,$x = 0$でも微分可能である.
\end{epl}


微分できる例をみたので,逆に微分できない例をみる.


\begin{epl}
$f(x) = |x|$
これは$x=0$の時,微分できない
\end{epl}


もう一つ例をみる


\begin{epl}
$f(x)$ = $
\begin{cases}
x & (x \ge 0)\\
x^2 & (x < 0)
\end{cases}
$は微分不可能.
実際に$g(x): = \frac{f(x)}{x}$とすると,$x < 0$では,$x^2$となり,$x \ge 0$では,1となるので,$0$での極限を考えると,収束しないことがわかる.
\end{epl}


\begin{rem}
最近は微積分は工学、例えば機械学習等でもよく使われるが,こうした世界で微分を扱う時は実際には有限個の点しか観測できないことともあり,微分不可能な点を無視することがある.
また,明らかに微分ができない場合でも劣微分という概念を用いて実質同様の議論ができる場合もある.
\end{rem}


上の例を見ると,微分ができるためには,直感的に"滑らか"でないといけないと想像するかもしれない.
その直感を補強する命題を紹介する.


\begin{prop}
$f$が$(a,b)$上微分可能な時,$f$は$(a,b)$上連続となる.
\end{prop}


\begin{proof}
$f(x) - f(a) = (x-a) \frac{f(x) - f(a)}{x -a}$より,$\lim_{x \to a}f(x) - f(a) = 0$となる.
\end{proof}


\begin{thm}[合成関数の微分]
  $g(x)$が微分可能であり.$f(x)$が微分可能であれば,

  \begin{equation*}
    \frac{f\circ g (x) - f\circ g (a)}{x-a} = f'(g(a))g'(a)
  \end{equation*}

となる.
\end{thm}

\begin{proof}
$a$の近傍$U$上で$g(x)$が定数とする.その場合,$g'(a) =0$となるが,
$\frac{f\circ g (x) - f\circ g (a)}{x-a} = 0$と一致するため,成り立つ.
$a$を含む$(a = \epsilon, a + \epsilon)$上で$g(x) != 0$とする.
$\frac{f(g(x)) - f(g(a))}{x- a} = \frac{f(g(x)-f(g(a))}{(g(x)-g(a))} \frac{(g(x)-g(a))}{x-a}$から計算できる.
で$h \to 0$の時$g(a+h) - g(a) \to 0$なので,問題ない。
$f'(g(a)) g'(a)$となる.
\end{proof}


\begin{dfn}
$f(x)$が一対一の関数のとき,$f(x)$の逆関数$g(x)$とは
$y = f(x)$に対し,$y \mapsto x$となる関数
\end{dfn}


\begin{epl}[指数関数と対数関数]
$3 \to 2^3$とすると$2^3 \to 3$となる関数.

この時$g(y+h) - g(y)/h$とする。
この時,$f(x) = y,f(x+l) = y+h$とすると
$h = f(x+l) - f(x)$であり$g(y+h) = x +l,g(y) = x$となる.

これより$x+l - l /f(x+l) - f(x)$となるので,
$1/f'(x)$となる.ただし$f(x) = y$となる.
\end{epl}


\begin{prop}[三角関数の微分]
$1-\cos x/x^2 = \lim (1 - \cos x)(1 + \cos x)/x^2 ...$

$sin'x =cosx$
$cos'x = - sinx$
$tan'x = 1/ cos^2x$
\end{prop}


\begin{prop}[対数関数の微分]
$\log_a(x+h) - \log_a(x)/h = \log_a(1+h/x)/h$
h/x=tとすると $ 1/x \log_a(1+t)/t$
となる.なのでこの値次第.
\end{prop}


実は
$\lim_{h \to 0} (1 +h)^{1/h}:=e$という値になる.
eの存在証明は飛ばすんですが、与太話をする。

\begin{rem}
こうした未知の概念んが出てきた時はそのイメージを掴むために適切な疑問を持つことが重要である.
例えばあたらいし概念に出会った時に最低限気にすることは以下である.
\begin{itemize}
  \item 定義として矛盾していないか
  \item 存在するか
  \item 存在ならば一つか?
  \item 存在する場所は制限できるか?
  \item 簡単に定理が成り立つ例が見つかるか?
  \item どういう性質を持つか.
\end{itemize}
\end{rem}


\begin{prop}[指数関数の微分]
逆関数の微分を使い
指数関数の微分をする.
$y= a^x$,$y= log_ax$
$dx/dy = 1/y \log_ae$なので,$a^x/\log_ae$が求める値になる.
logy = alogx
\end{prop}

これで初等的な関数の微分は全て終わる.

\section{関数の近似}
上の章では具体的に記述できる関数とそれらの合成や積で表せる関数に対して,具体的に計算できることを記述した.
ここではそのような関数の具体的な形を記述せずともわかる性質について調べる.
特に関数の元の値と微分の値について関係を調べ,
この章の終わりでは関数を多項式で近似する話を説明する.

\begin{thm}[ロルの定理]
関数$f: \mathbb{R} \to \mathbb{R}$が$[a, b]$上連続で$(a, b)$上連続とする.さらに$f(a) = f(b)$であれば,$f'(c) = 0$となる$c \in (a,b)$が存在する.
\end{thm}
非常に重要な定理である

\begin{thm}[平均値の定理]

\end{thm}

\begin{thm}[テイラーの定理]

\end{thm}

\begin{thm}[ロピタルの定理]

\end{thm}

\section{関数列の極限}