\part{一変数の微分}
ここでは,一変数関数の微分に関する基本的な性質を述べる.
定義から始まり、基本的な性質を調べる.
代数的な性質,位相的な性質,合成関数の微分法,積の微分法を説明し,
それらを元に初等的な関数の微分を記述する.
\section{微分の定義と初等的な性質}

\begin{screen}
\begin{dfn}[微分]
ある関数$f:U \to \mathbb{R}$に対し,
$x=a$で\textbf{微分可能}とは,
\begin{equation*}
\lim \frac{f(x) - f(a)}{x-a}
\end{equation*}
が存在することである.
上の極限を$f'(a)$とかき,$f(x)$の$x=a$での\textbf{微分}という.
極限の存在はある値$b$が存在し,$\forall \epsilon > 0$に対しある$\delta > 0$をうまく取ると$|x - a| < \delta$となる全ての$x$に対し,$|\frac{f(x) - f(a)}{x-a} - b| < \epsilon$が成り立つことを意味する.
\end{dfn}
\end{screen}


上の定義は関数が一点で微分可能かどうかを定義したものである.
つまり,上の定義では関数と点の組に対して,微分可能かを判定させる.
では関数に対して微分可能を定義するにはどうすればいいか.
一つの自然な答えは定義域の全ての点で微分可能である.
これを具体的に記述する.
\begin{dfn}
関数$f$が定義域$U$上の任意の点で微分可能な時,$f$を\textbf{微分可能}という.
\end{dfn}


実際にどのような関数が上の定義を満たしているかをみる.


\begin{epl}
以下の関数は微分可能である.
\begin{equation*}
f(x) = x^n (n \ge 0)
\end{equation*}
実際に微分の定義に従い,極限を求める.
\begin{equation*}
\frac{f(x) - f(a)}{x-a}= \sum_{i=0}^{n-1} x^{n-1-i}a^i
\end{equation*}
より,$g(x):=\frac{f(x) - f(a)}{x-a}$とすると,$g(x)$は連続関数であり,$g(a) = na^{n-1}$となる.
よって
\begin{equation*}
\lim_{x \to a} g(x) = na^{n-1}
\end{equation*}
となり,微分可能である.
\end{epl}


もう一つ微分可能な例をみる.

\begin{epl}
$f(x)$ = $
\begin{cases}
x^3 & (x \ge 0)\\
x^2 & (x < 0)
\end{cases}
$は微分可能.
上の例から,$x \neq 0$では微分可能である.
$a = 0$では

\begin{equation*}
g(x) := \frac{f(x) -f(0)}{x - 0} =
\begin{cases}
x^2 & (x \ge 0)\\
x & (x < 0)
\end{cases}
\end{equation*}

となるので,$\displaystyle \lim_{x \to 0} g(x) = 0$となり,$x = 0$でも微分可能である.
\end{epl}


微分できる例をみたので,逆に微分できない例をみる.


\begin{epl}
$f(x) = |x|$
これは$x=0$の時,微分できない
\end{epl}


もう一つ例をみる


\begin{epl}
$f(x)$ = $
\begin{cases}
x & (x \ge 0)\\
x^2 & (x < 0)
\end{cases}
$は微分不可能.
実際に$g(x): = \frac{f(x)}{x}$とすると,$x < 0$では,$x^2$となり,$x \ge 0$では,1となるので,$0$での極限を考えると,収束しないことがわかる.
\end{epl}


\begin{rem}
最近は微積分は工学、例えば機械学習等でもよく使われるが,こうした世界で微分を扱う時は実際には有限個の点しか観測できないことともあり,微分不可能な点を無視することがある.
また,明らかに微分ができない場合でも劣微分という概念を用いて実質同様の議論ができる場合もある.
\end{rem}


上の例を見ると,微分ができるためには,直感的に"滑らか"でないといけないと想像するかもしれない.
その直感を補強する命題を紹介する.


\begin{prop}
$f$が$(a,b)$上微分可能な時,$f$は$(a,b)$上連続となる.
\end{prop}


\begin{proof}
$f(x) - f(a) = (x-a) \frac{f(x) - f(a)}{x -a}$より,$\lim_{x \to a}f(x) - f(a) = 0$となる.
\end{proof}

\begin{thm}
\begin{itemize}
  \item $a \in \mathbb{R}$に対し,$af(x)$は微分可能
  \item $f(x) + g(x)$は微分可能
\end{itemize}
\end{thm}

\begin{thm}
  $f(x), g(x)$が微分可能な時,$f(x)\cdot g(x)$は微分可能.
\end{thm}
$f(x+h)g(x+h) - f(x)g(x) = f(x+h)g(x+h) - f(x)g(x+h) + f(x)g(x+h) - f(x)g(x)$と変形する.
\begin{equation*}
\frac{(x+h)g(x+h) - f(x)g(x)}{h} = \frac{f(x+h)g(x+h) - f(x)g(x+h)}{h} + \frac{f(x)g(x+h) - f(x)g(x)}{h}
\end{equation*}
となり,$\frac{f(x+h)g(x+h) - f(x)g(x+h)}{h}, \frac{f(x)g(x+h) - f(x)g(x)}{h}$は$h \to 0$の時に収束するので,
よって,
\begin{equation*}
  (f \cdot g)' = f'g + g f'
\end{equation*}
となる.

\begin{thm}[合成関数の微分]
  $g(x)$が微分可能であり.$f(x)$が微分可能であれば,
  \begin{equation*}
    \frac{f\circ g (x) - f\circ g (a)}{x-a} = f'(g(a))g'(a)
  \end{equation*}
となる.
\end{thm}

\begin{proof}
$a$の近傍$U$上で$g(x)$が定数とする.その場合,$g'(a) =0$となるが,
$\frac{f\circ g (x) - f\circ g (a)}{x-a} = 0$と一致するため,成り立つ.
$a$を含む$(a = \epsilon, a + \epsilon)$上で$g(x) != 0$とする.
$\frac{f(g(x)) - f(g(a))}{x- a} = \frac{f(g(x)-f(g(a))}{(g(x)-g(a))} \frac{(g(x)-g(a))}{x-a}$から計算できる.
で$h \to 0$の時$g(a+h) - g(a) \to 0$なので,問題ない。
$f'(g(a)) g'(a)$となる.
\end{proof}


\begin{dfn}
$f(x)$が一対一の関数のとき,$f(x)$の逆関数$g(x)$とは
$y = f(x)$に対し,$y \mapsto x$となる関数
\end{dfn}



\section{関数の近似}
上の章では具体的に記述できる関数とそれらの合成や積で表せる関数に対して,具体的に計算できることを記述した.
ここではそのような関数の具体的な形を記述せずともわかる性質について調べる.
特に関数の元の値と微分の値について関係を調べ,
この章の終わりでは関数を多項式で近似する話を説明する.

\begin{thm}[ロルの定理]
関数$f: \mathbb{R} \to \mathbb{R}$が$[a, b]$上連続で$(a, b)$上連続とする.さらに$f(a) = f(b)$であれば,$f'(c) = 0$となる$c \in (a,b)$が存在する.
\end{thm}
$f$が定数関数であれば,明らかに存在するので,$f$が定数関数でない時に示す.
この時$x \in (a,b)$で$f(x) \neq f(a)$となるものが存在する.
今$f(x) > f(a)$とする.
$[a, b]$上連続な関数はある点$c \in [a, b]$で最大値が存在するが,$f(x) > f(a)$より,$c \in (a, b)$となる.
よって,この点では微分可能であり,最大値を取るため,$f'(C) = 0$となる.
$f(x) < f(a)$の時も同様に示せる.

\begin{thm}[平均値の定理]
関数$f: \mathbb{R} \to \mathbb{R}$が$[a, b]$上連続で$(a, b)$上連続とする.
この時,以下を満たす$(c \in (a, b)$が存在する.
\begin{equation*}
  \frac{f(a) - f(b)}{a-b} = f'(c)
\end{equation*}
\end{thm}
\begin{equation*}
\phi(t) = f(t) -  \frac{f(a) - f(b)}{a- b}(t - b)
\end{equation*}
とする.この時,$\phi(a) = f(b), \phi(b) = f(b)$で,$[a, b]$上連続,$(a, b)$上微分可能なので,
ロルの定理を使うと
ある$c \in (a, b)$が存在し,$\phi'(c) = 0$となる,
これを変形すると,
\begin{equation*}
  \frac{f(a) - f(b)}{a-b} = f'(c)
\end{equation*}
となる.

\begin{rem}
  平均値の定理は多変数関数では成り立つとは限らない.
  それは$f_i(b) - f_i(a) = (f'_i(c_i))(b-a)$となるが$i$毎に$c_i$が異なるため,まとめてかけるとは限らないからである.
\end{rem}

しかし,以下の定理は多変数であっても成り立つ.
\begin{thm}
  以下は同値である.
  $f:I \to \mathbb{R}^n$で$I$の内側で全て微分可能であり,$I$上連続する時,
  \begin{itemize}
    \item $f'(x) = 0$となる.
    \item $f(x)$は定数関数.
  \end{itemize}
\end{thm}
\begin{proof}
$f'(x) = 0$であれば
平均値の定理より,$f(t) - f(a) = f'(x)$となるが,$f'(x) = 0$より,$f(x) = f(a)$となる.
逆は明らか.
\end{proof}

上の結果でもわかるように平均値の定理は2つのものを近似する時にその間のずれが発生する.
しかし,うまく工夫をすればそれを回避できる場合がある.

\begin{thm}[Cauchyの平均値の定理]
$g(b) \neq g(a)$で$\forall x \in (a,b)$上で$f'(x) \neq 0$か$g'(x) \neq 0$が成り立つとする.
この時,ある$c \in (a,b)$が存在し,
\begin{equation*}
  \frac{f(x) - f(a)}{g(x) - g(a)} = \frac{f'(c)}{g'(c)}
\end{equation*}
となる.
\end{thm}
\begin{proof}
$\phi(x) = (g(b) - g(a))f(x) - (f(b) - f(a)g(x)$とする.
この時,$\phi(a) = f(a)g(b) - f(b)g(a)$,$\phi(b) = -f(b)g(a) + f(a)g(b)$となるので,ロルの定理より
ある$c \in (a,b)$で,$(g(b) - g(a))f'(c) - (f(b) - f(a))g'(c) = 0$となる.
今$f'(c),g'(c)$のいずれかは0でないため,$g'(c) = 0$とすると,$g(a) \neq g(b)$と合わせると$f'(c) = 0$となるので,
$g'(c) \neq 0$となることがわかる.
よって,両辺を$g'(c)(g(b) - g(a)$で割ることで求める等式が得られた.
\end{proof}



上の定理は関数の差を微分を使って評価した.
これは変形すると$f(a) = f'(c)(a-b) + f(b)$となる.
$c \in (a,b)$より, $c = b + t(a-b)$とかける.
これより,$f(a) = f'(b + t(a-b))(a-b) + f(b)$となる.
これは高次元に一般化する.


\begin{thm}[テイラーの定理]
$f$が$[a, b]$で$n$階微分可能な時
\begin{equation*}
  f(x) = f(a) + \frac{f'(a)}{1!}(x-a) + \frac{f''(a)}{2!}(x-a)^2 + \cdots + \frac{f^{(n-1)}(a)}{n-1!}(x-a)^{n-1} + R_n(x)
\end{equation*}
とした時ある$c \in I$が存在し,
\begin{equation*}
R_n(x) = \frac{f^{(n)}(c)}{n!}(x-a)^n
\end{equation*}
とかける.($c$は$x$毎に変わることに注意せよ.)
\end{thm}
\begin{proof}
$ \phi(x) = f(x) - f(a) + \frac{f'(a)}{1!}(x-a) + \frac{f''(a)}{2!}(x-a)^2 + \cdots + \frac{f^{(n-1)}(a)}{n-1!}(x-a)^{n-1}$とする.
$\phi(a) = \phi'(a) = \ldots = \phi^{(n-1)}(a) = 0$となる.Cauchyの平均値の定理より
\begin{equation*}
\frac{\phi(x) - \phi(a)}{(x-a)^n} = \frac{\phi'(x_1) - \phi'(a)}{n(x_1 -a)^{n-1}} = \frac{\phi^{n}(x_n)}{n!}
\end{equation*}
となるので,これを変形することで示せる.
\end{proof}

さらにTaylorの剰余項は以下のように表現できる.
\begin{prop}
$f$が$C^{n}$級の時,
\begin{equation*}
R_n(x) = \int \frac{(x-a)^{n-1}}{(n-1)!}f^{(n)}(t)dt
\end{equation*}
となる.
\end{prop}
このノートではまだ積分について定義していないので,演習問題とする.
$C^n$級についても多変数の時に定義する.
$C^n$から被積分関数が可積分であることがわかり,部分積分で変形することで帰納的にわかる.

コーシーの平均値の定理から極限を取ると
\begin{thm}[ロピタルの定理]
$f(x),g(x)$が微分可能かつ$g(x) \neq g(a)$かつ,$g'(a) \neq 0$とすると,以下が成り立つ.
\begin{equation*}
  \lim_{x \to a}\frac{f(x)-f(a)}{g(x) - g(a)} = \frac{f'(a)}{g'(a)}
\end{equation*}
\end{thm}
\begin{rem}
 ロピタルの定理は実際にはいくつかの亜種があり,例えば右辺は$a$での値が存在しなくとも,極限で一致していればよい.
\end{rem}

\section{関数列の極限}
微分可能か関数の数列の極限を用いて新しく関数を作ろうとする時,その関数にどういう性質を認めるか考えたい.
